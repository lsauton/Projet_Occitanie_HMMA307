\documentclass[a4paper,11pt]{article}


\newcommand{\titrecours}{Classifieur de Bayes}
\newcommand{\titretd}{Apprentissage statistique}
\newcommand{\auteur}{Joseph Salmon}
\newcommand{\annee}{Master M2MO: 2017/2018}
\newcommand{\ecole}{Universit\'e Paris Diderot}% \titre{}

\usepackage{../sty/td}
\usepackage{../sty/shortcuts_js}
\addbibresource{../biblio/references_all.bib}


\begin{document}
\sloppy
\feuille{1}

\medskip


% DGL p.11-12
\exercice  On considère un modèle de classification où le couple aléatoire $(X, Y)$ est de loi $P$ décrite par :
\begin{itemize}
\item la loi de $X$ est une loi de probabilité $P_X$ sur $\bbR$
\item la fonction de régression $\displaystyle \eta(x) = \frac{x}{x+\theta}$ où $\theta>0$ fixé.
\end{itemize}
On note $h^*$ le classifieur de Bayes. Expliciter le classifieur de Bayes dans ce modèle. Montrer ensuite que son risque ``0-1'' vaut
\begin{equation*}
	R(h^*)=\int \min(\eta(x),1-\eta(x)) dP_X(x).
\end{equation*}
 Calculer le risque de Bayes lorsque $P_X = \cU([0, \alpha\theta])$ où $\alpha>1$.



%%%%%%%%%%%%%%%%%%%%%%%%%%%%%%%%%%%%%%%%%%%%%%%%%%%%%%%%%%%%%%%%%%%%%%%%%%%%%%%
%%%%%%%%%%%%%%%%%%%%%%%%%%%%%%%%%%%%%%%%%%%%%%%%%%%%%%%%%%%%%%%%%%%%%%%%%%%%%%%
\section*{Conseils bibliographiques}
\label{sec:bibliographie}
%%%%%%%%%%%%%%%%%%%%%%%%%%%%%%%%%%%%%%%%%%%%%%%%%%%%%%%%%%%%%%%%%%%%%%%%%%%%%%%
%%%%%%%%%%%%%%%%%%%%%%%%%%%%%%%%%%%%%%%%%%%%%%%%%%%%%%%%%%%%%%%%%%%%%%%%%%%%%%%


Vous trouverez ci-dessous quelques points d’entrée utiles pour l'apprentissage automatique:

\begin{itemize}
	\item Théorique et porté sur les aspects probabilistes: \cite{Devroye_Gyorfi_Lugosi96}
	\item Utilitaire et porté sur les aspects pratiques: \cite{Hastie_Tibshirani_Friedman09}
	\item Livre récent porté essentiellement sur l’aspect optimisation: \cite{Shalev-Shwartz_Ben-David14} (et du même auteur sur l'apprentissage en ligne \cite{Shalev-Shwartz11})
	\item Méthodes Bayésiennes et modèles graphiques: \cite{Murphy12}
\end{itemize}



%%%%%%%%%%%%%%%%%%%%%%%%%%%%%%%%%%%%%%%%%%%%%%%%%%%%%%%%%%%%%%%%%%%%%%%%%%%%%%%
\printbibliography
 %%%%%%%%%%%%%%%%%%%%%%%%%%%%%%%%%%%%%%%%%%%%%%%%%%%%%%%%%%%%%%%%%%%%%%%%%%%%%%




\end{document}
